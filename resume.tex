\documentclass{resume}
\usepackage{zh_CN-Adobefonts_external} % 中文字体支持
\usepackage{linespacing_fix}
\usepackage{cite}
\usepackage{comment}

\usepackage[Chinese]{languageSelection}
\usepackage[color=blue]{notePlus}

\begin{document}
\pagenumbering{gobble}

\CN{
  \name{黄锡超}
  \info{手机: (+86) 13736707307}{邮箱: hsn.zj@foxmail.com}{}{}
  \info{性别: 男}{微信: hsn7307}{在线简历: resume.huangsn.dev}{}
}

% =============================
% 教育背景
% =============================
\CN{
  \section{教育背景}
  \datedsubsection{\textbf{浙江理工大学科技与艺术学院},计算机科学与技术(本科)}{2021.09 -- 2025.06}
}

% =============================
% 工作经历
% =============================
\CN{
  \section{工作经历}

  \datedsubsection{\textbf{智识无垠},后端开发工程师}{2025.06 -- 至今}
  \begin{itemize}[leftmargin=*, itemsep=0.5ex]
    \item 参与核心文档解析平台的后端架构演进,支撑 \textbf{万级 DAU、千万级日解析量},服务可用性稳定。
    \item 主导内部平台建设,推动关键流程 \textbf{自动化与标准化},运营效率提升 \textbf{2 倍以上}。
  \end{itemize}

  \datedsubsection{\textbf{谐云},后端开发工程师}{2024.07 -- 2025.04}
  \begin{itemize}[leftmargin=*, itemsep=0.5ex]
    \item 负责 PaaS 平台核心服务的 \textbf{Java→Go 重构} 与性能优化,使服务平均响应时间降低 \textbf{40\%+}。
    \item 参与多租户、容器服务等模块建设,优化资源调度与系统稳定性,显著提升用户体验。
  \end{itemize}
}

% =============================
% 项目经历
% =============================
\CN{
  \section{项目经历}

  % -------- Doc2X --------
 \datedsubsection{\textbf{Doc2X — C/B 端智能文档解析平台},后端开发}{2025.06 -- 至今}
\begin{itemize}[leftmargin=*, itemsep=0.6ex]
  \item \textbf{项目简介}:行业领先的文档解析平台,支持从 PDF 智能提取内容并转换为 \textbf{Markdown / LaTeX / Word},覆盖学术论文、合同、图表等场景,日解析量超 \textbf{千万级}。
  \item \textbf{技术栈}:\textbf{Go}、\textbf{gRPC / ConnectRPC}、\textbf{PostgreSQL}、\textbf{Redis}、\textbf{NATS}、\textbf{Kubernetes}、\textbf{OpenTelemetry}
  \item \textbf{核心贡献}:
  \begin{itemize}[leftmargin=2em, itemsep=0.4ex]
    \item \textbf{多渠道通知系统}:基于 \textbf{策略模式 + 工厂模式} 设计\textbf{可扩展通知框架},新渠道接入时间从 \textbf{2 天缩短至 0.5 天}。
    \item \textbf{统一消费者服务}:基于 \textbf{NATS} 构建事件链路,实现 \textbf{额度扣减、限流、幂等校验、缓存更新} 等核心处理流程。
    \item \textbf{大模型翻译}:基于 \textbf{Prefill 优化} 请求结构,提升 \textbf{整体 KV Cache 命中率};整体 \textbf{Token 消耗降低 20\%~40\%},显著减少推理成本。
    \item \textbf{管理后台开发}:使用 \textbf{Vibe Coding} 完成公告管理、发票/退款审批、订单查询等功能全栈开发,推进运营流程\textbf{数字化},显著提升内部效率。
  \end{itemize}
\end{itemize}

% -------- 观云台 --------
\datedsubsection{\textbf{观云台 — Kubernetes PaaS 平台},后端开发}{2024.07 -- 2025.05}
\begin{itemize}[leftmargin=*, itemsep=0.6ex]
  \item \textbf{项目简介}:基于 \textbf{Kubernetes} 的企业级 PaaS,帮助企业降低运维成本并提升业务交付效率。
  \item \textbf{技术栈}:\textbf{Go}、\textbf{Gin}、\textbf{Kubernetes}、\textbf{Kubebuilder}、\textbf{MySQL}
  \item \textbf{核心贡献}:
  \begin{itemize}[leftmargin=2em, itemsep=0.4ex]
    \item \textbf{Kubernetes 资源管理重构}:完成 \textbf{Java→Go 重构},基于 \textbf{Kubernetes API 规范} 引入\textbf{原生 RBAC 鉴权};修复 \textbf{StatefulSet 字段约束}导致的部署死锁问题,显著提升扩容 / 发布稳定性。
    \item \textbf{多租户备份性能优化}:引入 \textbf{MapReduce 思想} 与 \textbf{可配置协程池},将串行调用改为 \textbf{分片并行},显著减少任务总时长并沉淀为通用工具。
    \item \textbf{容器服务(网关)重构}:将命令式接口切换为 \textbf{Kubernetes 声明式架构},核心操作延迟从 \textbf{20 秒 → 300 毫秒},显著提升用户体验。
    \item \textbf{基线合规巡检}:基于 \textbf{K8S CronJob} 实现定时巡检,覆盖 \textbf{文件 + 命令} 两类检查,确保纳管组件版本 \textbf{100\% 合规}。
  \end{itemize}
\end{itemize}

}

% =============================
% 技能
% =============================
\CN{
  \section{专业技能}
  \begin{itemize}[leftmargin=*, itemsep=0.5ex]
    \item \textbf{编程语言与框架}:熟练使用 Go;掌握 gRPC、ConnectRPC、Gin;具备微服务架构设计经验。
    \item \textbf{云原生技术}:熟悉 Kubernetes 核心机制,具备通过源码分析定位问题的经验。
    \item \textbf{数据库与缓存}:熟练 PostgreSQL 设计与优化;掌握 Redis 在高并发场景下(击穿/雪崩/穿透)的解决方案。
    \item \textbf{消息队列与事件驱动}:熟练使用 NATS,具备事件驱动架构设计与异步处理经验。
    \item \textbf{AI 与大模型}:了解大模型原理,具备 MCP 相关开源经验。
  \end{itemize}
}

\end{document}
