\documentclass{resume}
\usepackage{zh_CN-Adobefonts_external} % 中文字体支持
\usepackage{linespacing_fix}           % 行间距修正
\usepackage{cite}
\usepackage{comment}

\usepackage[Chinese]{languageSelection}
\usepackage[color=blue]{notePlus}

\begin{document}
\pagenumbering{gobble}
\CN{
  \name{黄锡超}
  \info{手机:(+86) 13736707307}{邮箱:hsn.zj@foxmail.com}{}{}
  \info{性别:男}{微信:hsn7307}{在线简历:resume.huangsn.dev}{}
}

\CN{
  \section{教育背景}
  \datedsubsection{\textbf{浙江理工大学科技与艺术学院},计算机科学与技术,\textbf{本科}}{2021.09 - 2025.06}
}
 \CN{
    \section{专业技能}
    \begin{itemize}[leftmargin=*, parsep=0.5ex, itemsep=0.5ex]
      \item \textbf{编程语言与框架}:\textbf{熟练使用 Go},掌握 \textbf{gRPC、ConnectRPC、Gin} 等主流框架,具备\textbf{微服务架构设计}经验。
      \item \textbf{云原生技术}:\textbf{熟悉 Kubernetes 核心机制},能通过\textbf{源码分析}定位并解决过问题。
      \item \textbf{数据库与缓存}:\textbf{熟练 PostgreSQL 设计与优化},具备 \textbf{Redis 在高并发场景下的实战经验}(缓存击穿/雪崩/穿透)。
      \item \textbf{消息队列与事件驱动}:熟练使用 \textbf{NATS},具备\textbf{事件驱动架构}设计经验,掌握\textbf{异步消息处理}。
      \item \textbf{工程与部署}:熟悉 \textbf{Linux} 与 \textbf{Git},具备\textbf{从开发到部署的全流程交付能力},熟悉 \textbf{OpenTelemetry} 链路追踪。
      \item \textbf{AI 与大模型}:了解大模型原理,具备 \textbf{MCP 相关开发经验}。
    \end{itemize}
  }

\CN{
  \section{工作经历}
  \datedsubsection{\textbf{智识无垠},\textbf{后端开发工程师}}{2025.06 - 至今}
  \begin{itemize}[leftmargin=*, parsep=0.5ex, itemsep=0.5ex]
    \item 参与核心文档解析产品后端系统的\textbf{架构设计与迭代},支撑 \textbf{万级日活、千万级日文档解析量} 的稳定运行。
    \item 推动\textbf{内部平台建设}与\textbf{流程自动化},支撑产品的快速迭代。
    \item 与算法团队紧密协作,推动\textbf{模型能力的工程化落地},提升文档解析的准确率与稳定性。
  \end{itemize}

  \datedsubsection{\textbf{谐云},\textbf{后端开发工程师}}{2024.07 - 2025.04}
  \begin{itemize}[leftmargin=*, parsep=0.5ex, itemsep=0.5ex]
    \item 参与 PaaS 云平台\textbf{核心服务的重构与迭代},推动系统\textbf{现代化改造}与\textbf{性能优化}。
    \item 协助构建\textbf{多租户、容器服务}等模块,提升平台\textbf{稳定性与用户体验}。
  \end{itemize}
}
\CN{
  \section{项目经历}
  \datedsubsection{\textbf{Doc2X — C/B端文档解析平台},后端开发}{2025.06 - 至今}
    \begin{itemize}[parsep=0.5ex]
      \item \textbf{项目简介}:业界领先的\textbf{智能文档解析器},支持无损提取 PDF 内容并转化为 Markdown/LaTeX/Word 等格式,服务日均解析量超 \textbf{千万级},涵盖学术论文、合同文档、表格图表等多场景解析。
      \item \textbf{技术栈}:\textbf{Go}、\textbf{gRPC + ConnectRPC}、\textbf{PostgreSQL}、\textbf{Redis}、\textbf{NATS JetStream}、\textbf{Kubernetes}、\textbf{OpenTelemetry}。
      \item \textbf{贡献亮点}:
        \begin{itemize}[leftmargin=*, parsep=0pt, itemsep=0.5ex]
          \item {\textbf{微服务架构设计}}:构建\textbf{事件驱动的异步微服务架构},通过 gRPC 服务间通信和 NATS JetStream 消息队列,将业务逻辑与算法服务解耦。
          \item {\textbf{多渠道通知系统}}:设计并实现\textbf{告警通知系统},采用\textbf{策略模式 + 工厂模式}接入短信、邮件、飞书等渠道,支持余额告警、资源告警、系统公告等场景,通过\textbf{智能渠道选择}和\textbf{频次限流机制}确保通知送达率。
          \item {\textbf{算法服务协作}}:与算法团队协作推进\textbf{OCR/Layout 模型集成},设计异步处理流水线,优化文档解析链路的\textbf{稳定性和准确率},支持大规模并发解析请求。
        \end{itemize}
    \end{itemize}
 \datedsubsection{\textbf{观云台 — Kubernetes PaaS 平台},后端开发}{2024.07 - 2025.05}
 \begin{itemize}[parsep=0.5ex]
   \item \textbf{项目简介}:基于 Kubernetes 的\textbf{新一代 PaaS 平台},帮助企业提升 IT 管理能力,降低运维成本,保障业务稳定与高效迭代。
   \item \textbf{技术栈}:\textbf{Go}、\textbf{Gin}、\textbf{Kubernetes}、\textbf{Kubebuilder}、\textbf{MySQL}。
   \item \textbf{贡献亮点}:
     \begin{itemize}[leftmargin=*, parsep=0pt, itemsep=0.5ex]
       \item \textbf{资源管理重构}:将Kubernetes资源服务管理从 \textbf{Java 重构为 Go},接口参照 \textbf{Kubernetes 规范}并集成鉴权,\textbf{增强平台安全性与一致性}。
       \item \textbf{性能优化}:独立实现\textbf{多租户备份服务},结合 \textbf{MapReduce 思想}与\textbf{协程优化},\textbf{响应时间从 30 秒缩短至 2 秒},并沉淀为可复用工具。
       \item \textbf{架构升级}:将 Java 命令式容器服务改造为 \textbf{Kubernetes 声明式架构},统一资源模型,\textbf{核心接口响应时间从 20 秒降至 300 毫秒},显著改善用户体验。
       \item \textbf{基线合规检查}:为保障售后服务资格,与基础设施部门合作建立\textbf{自动化合规巡检机制}。利用 \textbf{Kubernetes CronJob} 实现定时巡检,支持\textbf{文件和命令}两种检查模式,确保纳管组件版本 100\% 符合规范。
     \end{itemize}
 \end{itemize}
}
\CN{
  % \section{开源经历}
  % \datedsubsection{\textbf{tiny-redis (Go)}}{11 Stars}
  % \begin{itemize}[parsep=0.5ex]
  %   \item 实现一个支持多种数据结构与 TTL 的\textbf{内存数据库},熟悉底层\textbf{网络库}与\textbf{存储实现}。
  % \end{itemize}

  % \datedsubsection{\textbf{Kubernetes-mcp (Go)}}{25 Stars}
  % \begin{itemize}[parsep=0.5ex]
  %   \item 实现 \textbf{Model Context Protocol (MCP)} 服务器,支持与 Kubernetes 集群交互,实现\textbf{高效资源管理与调度}。
  % \end{itemize}
}
\end{document}
