\documentclass{resume}
\usepackage{zh_CN-Adobefonts_external} % 中文字体支持
\usepackage{linespacing_fix}
\usepackage{cite}
\usepackage{comment}

\usepackage[Chinese]{languageSelection}
\usepackage[color=blue]{notePlus}

\begin{document}
\pagenumbering{gobble}

\CN{
  \name{黄锡超}
  \info{手机: (+86) 13736707307}{邮箱: hsn.zj@foxmail.com}{}{}
  \info{性别: 男}{微信: hsn7307}{在线简历: resume.huangsn.dev}{}
}

% =============================
% 教育背景
% =============================
\CN{
  \section{教育背景}
  \datedsubsection{\textbf{浙江理工大学科技与艺术学院},计算机科学与技术(本科)}{2021.09 -- 2025.06}
}

% =============================
% 工作经历
% =============================
\CN{
  \section{工作经历}

  \datedsubsection{\textbf{智识无垠},后端开发工程师}{2025.06 -- 至今}
  \begin{itemize}[leftmargin=*, itemsep=0.5ex]
    \item 参与核心文档解析平台的后端架构演进,支撑 \textbf{十万级 DAU、千万级日解析量},服务可用性稳定。
    \item 负责内部平台建设,推动关键流程 \textbf{自动化与标准化},运营效率提升 \textbf{2 倍以上}。
  \end{itemize}

  \datedsubsection{\textbf{谐云},后端开发工程师}{2024.07 -- 2025.04}
  \begin{itemize}[leftmargin=*, itemsep=0.5ex]
    \item 负责 PaaS 平台容器服务的 \textbf{Java→Go 重构} 与性能优化,使服务平均响应时间降低 \textbf{40\%+}。
    \item 参与多租户、容器服务等模块建设,优化资源调度与系统稳定性,显著提升用户体验。
  \end{itemize}
}

% =============================
% 项目经历
% =============================
\CN{
  \section{项目经历}

  % -------- Doc2X --------
 \datedsubsection{\textbf{Doc2X — C/B 端智能文档解析平台},后端开发}{2025.06 -- 至今}
\begin{itemize}[leftmargin=*, itemsep=0.6ex]
  \item \textbf{项目简介}:行业领先的文档解析平台,支持从 PDF 智能提取内容并转换为 \textbf{Markdown / LaTeX / Word},覆盖学术论文、合同、图表等场景,日解析量超 \textbf{千万级}。
  \item \textbf{技术栈}:\textbf{Go}、\textbf{gRPC / ConnectRPC}、\textbf{PostgreSQL}、\textbf{Redis}、\textbf{NATS}、\textbf{Kubernetes}、\textbf{OpenTelemetry}
  \item \textbf{核心贡献}:
  \begin{itemize}[leftmargin=2em, itemsep=0.4ex]
    \item \textbf{核心服务高可用}:构建\textbf{分级限流与闲时排队}体系,确保超额流量下的系统稳定;针对解析/导出环节实施\textbf{缓存策略},保障高并发场景下的服务可用性。
    \item \textbf{多渠道通知}:基于\textbf{策略+工厂模式}抽象通知中台,统一接入短信、邮件及飞书/钉钉等主流平台,实现业务解耦与高度可扩展性。
    \item     \textbf{大模型翻译}:为了翻译服务\textbf{整体 KV Cache 命中率},基于 \textbf{Prefill 优化} 请求结构,整体 \textbf{Token 消耗降低 20\%~40\%}。
    \item     \textbf{管理后台开发}:运用 \textbf{Vibe Coding} 模式主导运营中台数字化建设,独立完成公告管理、审批流(发票/退款)、订单分析及兑换码等核心模块的全栈开发,显著提升了内部运营效率与业务支撑能力。
  \end{itemize}
\end{itemize}

% -------- 观云台 --------
\datedsubsection{\textbf{观云台 — Kubernetes PaaS 平台},后端开发}{2024.07 -- 2025.04}
\begin{itemize}[leftmargin=*, itemsep=0.6ex]
  \item \textbf{项目简介}:基于 \textbf{Kubernetes} 的企业级 PaaS,帮助企业降低运维成本并提升业务交付效率。
  \item \textbf{技术栈}:\textbf{Go}、\textbf{Gin}、\textbf{Kubernetes}、\textbf{Kubebuilder}、\textbf{MySQL}
  \item \textbf{核心贡献}:
  \begin{itemize}[leftmargin=2em, itemsep=0.4ex]
\item \textbf{资源管理重构}:主导 Java 至 Go 的核心组件重构,深度整合 \textbf{K8s 原生 RBAC 鉴权规范};修复 StatefulSet 字段约束导致的更新死锁问题,确保大规模集群发布与扩缩容的稳定性。
\item \textbf{网关架构演进}:推动容器网关由\textbf{命令式向声明式架构}转型,利用 Controller Reconcile机制}将核心操作响应由 \textbf{20s →300ms}。
\item \textbf{自动化合规巡检}:基于 CronJob + 自定义检测引擎 构建基线审计模块,实现全链路文件与配置一致性检查,保障纳管组件合规率达 \textbf{100\%}。
  \end{itemize}
\end{itemize}

}

% =============================
% 技能
% =============================
\CN{
  \section{专业技能}
  \begin{itemize}[leftmargin=*, itemsep=0.5ex]
    \item \textbf{编程语言与框架}:熟练使用 Go;掌握 gRPC、ConnectRPC、Gin;具备微服务架构设计经验。
    \item \textbf{云原生技术}:熟悉 Kubernetes 核心机制,具备通过源码分析定位问题的经验。
    \item \textbf{数据库与缓存}:熟练 PostgreSQL 设计与优化;掌握 Redis 在高并发场景下(击穿/雪崩/穿透)的解决方案。
    \item \textbf{消息队列与事件驱动}:熟练使用 NATS,具备事件驱动架构设计与异步处理经验。
    \item \textbf{AI 与大模型}:了解大模型原理,具备 MCP 相关开源经验。
  \end{itemize}
}

\end{document}
