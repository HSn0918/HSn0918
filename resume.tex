% !TeX encoding = UTF-8
% !TeX program = xelatex
% !TeX spellcheck = en_US
% --- 选择Xelatex编译器 ---
\documentclass{resume}

% --- 核心依赖包 ---
\usepackage{zh_CN-Adobefonts_external} % 中文字体支持
\usepackage{linespacing_fix}           % 行间距修正
\usepackage{cite}                      % 参考文献
\usepackage{comment}                   % 注释环境

% --- 推荐阅读 ---
% LaTeX 30分钟入门: https://www.overleaf.com/learn/latex/Learn_LaTeX_in_30_minutes
% languageSelection 包文档: https://github.com/xiangrongjingujiu/latex-languageSelection
% notePlus 包文档: https://github.com/xiangrongjingujiu/latex-note-plus

% --- 自定义功能包 ---
\usepackage[Chinese]{languageSelection} % 中英双语切换
\usepackage[color=blue]{notePlus}      % 笔记与高亮

\begin{document}
\pagenumbering{gobble} % 全局隐藏页码

% ===================================================================
% 基本信息
% 替换 "%" 后面的注释内容为你自己的信息
% ===================================================================

% --- 中文信息 ---
\CN{
  \name{黄锡超}
  \info{手机:(+86) 13736707307}{邮箱:hsn.zj@foxmail.com}{}{}
  \info{性别:男}{微信:hsn7307}{}{}
}


% --- 照片 ---
% 照片需放置于 images 文件夹下,并命名为 you.jpg
% 0.15 为照片缩放比例,可按需调整
% 如果不需要照片,请注释掉下面这行
\yourphoto{0.10}

% ===================================================================
% 正文内容
% ===================================================================

% -------------------------------------------------------------------
% 教育背景
% -------------------------------------------------------------------
\CN{
  \section{教育背景}
  \datedsubsection{\textbf{浙江理工大学科技与艺术学院},计算机科学与技术,\textit{本科}}{2021.09 - 2025.06}
}
\EN{
  \section{EDUCATION}
  \datedsubsection{\textbf{Clayden University}, Clayden Experimental Class, \textit{PhD}}{1910.09 - 1930.06}
}
\CN{
  \section{专业技能}
  \begin{itemize}[leftmargin=*, parsep=0.5ex, itemsep=0.5ex]
    \item \textbf{编程语言与框架}:熟悉Go语言(核心特性如map, channel, slice)及主流框架 (如gin, kubebuilder, connect-grpc)。
    \item \textbf{云原生技术}:熟悉Kubernetes核心机制(含Informer),具备通过源码分析解决业务问题的能力。
    \item \textbf{数据库与缓存}:熟练运用MySQL进行数据存储与管理;熟悉Redis并具备解决缓存穿透、雪崩、击穿等高并发问题的实践经验。
    \item \textbf{开发与部署}:熟悉Linux操作系统和Git版本控制;具备独立完成项目开发到部署全流程的能力。
    \item \textbf{AI与大模型}:了解AI大模型Prompt工程(提示词设计优化),熟悉MCP开发。
  \end{itemize}
}
% -------------------------------------------------------------------
% 工作经历
% -------------------------------------------------------------------
\CN{
  \section{工作经历}
  \datedsubsection{\textbf{武汉智识无垠科技有限公司},后端开发工程师}{2025.06 - 至今}
  \begin{itemize}[leftmargin=*, parsep=0.5ex, itemsep=0.5ex]
    \item 负责核心产品后端的架构设计、服务迭代与体验优化,负责内部平台的规划与研发工作。
    \item 与算法团队紧密协作,专注于提升系统在高并发场景下的稳定性与扩展性,保障核心算法服务的平稳落地与高效运行。
  \end{itemize}

  \datedsubsection{\textbf{杭州谐云科技有限公司},后端开发工程师}{2024.07 - 2025.04}
  \begin{itemize}[leftmargin=*, parsep=0.5ex, itemsep=0.5ex]
    \item 负责PaaS平台核心服务的现代化改造,主导后端服务的迁移与重构工作,并为新架构设计与构建标准化的API。
    \item 深入分析并解决系统的性能瓶颈,通过架构与代码层面的持续优化,实现关键服务响应速度与吞吐量的显著提升。
  \end{itemize}
}

% -------------------------------------------------------------------
% 项目经历
% -------------------------------------------------------------------
\CN{
  \section{项目经历}
    \datedsubsection{\textbf{Doc2X,后端开发工程师}}{2025.06-至今}
  \begin{itemize}[parsep=0.5ex]
    \item \textbf{项目介绍}:全场景文档解析器 Doc2X,提供业界领先的 PDF 解析服务,旨在无损提取并结构化文档内容,并一键转换为 Markdown, LaTeX, Word 等多种格式。
    \item \textbf{主要贡献}:
      \begin{itemize}[leftmargin=*, parsep=0pt, itemsep=0.5ex, topsep=0.5ex]
        \item \textbf{自动化工作流平台研发}:设计并开发内部平台,替代原有手动流程。
        \item \textbf{核心架构解耦与设计}:构建事件驱动的异步架构,将业务逻辑与算法服务解耦,显著提升了系统的可维护性与迭代效率。
      \end{itemize}
  \end{itemize}
}
  \datedsubsection{\textbf{观云台,后端开发工程师}}{2024.07-2025.05}
  \begin{itemize}[parsep=0.5ex]
    \item \textbf{项目介绍}:基于Kubernetes的新一代PaaS平台,旨在提升企业IT管理能力,降低运营成本和风险,提高运维效率,保障业务稳定与高效迭代。
    \item \textbf{主要贡献}:
      \begin{itemize}[leftmargin=*, parsep=0pt, itemsep=0.5ex, topsep=0.5ex]
        \item \textbf{平台重构与安全增强}:参与平台Kubernetes资源管理重构,将核心服务从 \textbf{Java} 迁移至 \textbf{Go},并参照Kubernetes规范设计接口,集成鉴权功能,显著增强平台安全性与一致性。
        \item \textbf{多租户管理与性能优化}:独立设计并实现备份服务器多租户功能,结合MapReduce思想和协程优化接口性能,将响应时间从30秒缩短至2秒,并通用化该工具方法,大幅提高项目开发效率。
        \item \textbf{容器服务架构改造与性能提升}:将Java命令式服务改造为K8S声明式架构,统一资源模型;通过协程与批处理优化,核心接口响应时间从20秒降至300毫秒,显著提升用户体验与系统效率。
      \end{itemize}
  \end{itemize}



\CN{
  \section{开源经历}
%  \datedsubsection{\textbf{tiny-redis} (Go)}{11 Stars}
%  \begin{itemize}[parsep=0.5ex]
%   \item \textbf{项目介绍}:基于Go网络库实现的一个支持多种数据结构 (以及TTL) 的内存数据库。
%  \end{itemize}
 
  \datedsubsection{\textbf{Kubernetes-mcp} (Go)}{23 Stars}
  \begin{itemize}[parsep=0.5ex]
    \item \textbf{项目介绍}:基于Go实现的 Model Context Protocol (MCP) 服务器,用于与Kubernetes集群交互,提供高效的集群管理和资源调度能力。
  \end{itemize}
}

\end{document}
